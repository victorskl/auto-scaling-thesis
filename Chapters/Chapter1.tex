% Chapter 1

\chapter{Introduction} % Main chapter title

\label{Chapter1} % For referencing the chapter elsewhere, use \ref{Chapter1} 

%----------------------------------------------------------------------------------------

% Define some commands to keep the formatting separated from the content 
\newcommand{\keyword}[1]{\textbf{#1}}
\newcommand{\tabhead}[1]{\textbf{#1}}
\newcommand{\code}[1]{\texttt{#1}}
\newcommand{\file}[1]{\texttt{\bfseries#1}}
\newcommand{\option}[1]{\texttt{\itshape#1}}

%----------------------------------------------------------------------------------------

%\section{Motivation}

\textbf{Motivation:} \quad Today, in software engineering, the rise of Containerisation \parencite{Scheepers2014VirtualizationAC} and Cloud Computing \parencite{Armbrust:2010:VCC:1721654.1721672} technologies have prominently changed the way we deliver  software packaging and deployment. Certainly, application deployment is not quite like it used to be -- concepts such as Containerised Application \parencite{Merkel:2014:DLL:2600239.2600241}, Infrastructure-as-Code (IaC), DevOps \parencite{httermann2012devops} are some examples in the modern deployment literature. Advances in virtualization technologies have changed the Operating System (OS) landscape and studies \parencite{Kozhirbayev2017APC} \parencite{5708625} show that container-based virtualizaton are the best fit for application process level isolation and, hypervisor-based full virtualization are better for infrastructure level islolation. Nevertheless, it is evident that the rise of virtualization technologies contributed the emergence of Cloud Computing. One profound benefit of Cloud Computing is giving virtually unlimited number of computing resources -- Virtual Machine (VM) and, the elasticity to acquire and release of these resources. Software development is also advancing in more decomposable and serviceable components such as adopting Service-Oriented Architecture (SOA), Micro-services and RESTful architecture. These advances are realisation of Distributed Systems -- where it defines as \textit{software components located at networked computers communicate and coordinate theirs actions only by passing messages} \parencite{Coulouris:2011:DSC:2029110}. Studies \parencite{Sinnott:2016:SCS:3008079.3008128} show that a \textbf{Scalable} cloud-based distributed application can be built and deployed it onto the Cloud platform. This also gives the indication that we can dynamically scale to meet with on-demands computing needs. The idea of \textbf{Auto-Scaling} system is to give business:

\begin{enumerate}
\item to start with the minimal operational computing resource provision at deployment
\item to scale out the system to meet the usage demands by probing particular resource metrics such as CPU utilisation
\item to spin down the computing resources to maintain the minimal operational limit when no usage demand is required
\item to automate this scaling process 1-3 without human operator intervention
\end{enumerate}

This elastic nature of dynamic scaling give operational optimisation which contribute business to streamline costs, increase productivity and improve profitability while maintaining the Service Level Objectives (SLO) and satisfying Quality of Service (QoS).
\\
\\
%\section{Aim and Objective}
\textbf{Research:} \quad To pursuit this motivation in real world setting, this project investigates the auto-scaling of ATHENA software stack on the Cloud platform. The investigation involves containerised application scaling using Docker Compose, Docker Swarm and Kubernetes orchestration.

% and, infrastructure level scaling using OpenStack.

%\section{Thesis Structure}

The remainder of the thesis is organised as follows. Chapter 2 explain an overview of ATHENA system and its domain driver. Chapter 3 discuss ATHENA deployment, containerisation and scaling using Docker. Chapter 4 discuss the insightful research on Kubernetes, its auto-scaling and finding. Chapter 5 discuss the related studies and reviews. Chapter 6 give concluding remarks and identify areas for future work.

